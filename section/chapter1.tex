\chapter{PENDAHULUAN}

\section{Latar Belakang}
Perkembangan Aplikasi pada saat ini telah mengalami peningkatan yang sanggat drastis \cite{neyfa2016perancangan}. Hal tersebut dapat didukung dengan adanya kemudahan aplikasi saat ini yang bermacam-macam salah satunya yaitu Downloader MP4 Youtube (savefrom.net). Aplikasi Downloader MP4 Youtube Merupakan aplikasi download video yang popular digunakan oleh pengguna komputer atau laptop. Selain pengguan komputer dan laptop, aplikasi Downloader Mp4 juga disediakan di android. Aplikasi Downloader Mp4 tidak hanya tersedia dalam bentuk aplikasi tetapi juga tersedia dalam bentuk web \cite{neyfa2016perancangan} . Aplikasi Downloader Mp4 ini digunakan sebagai media mendownload video tanpa adanya batasan ukuran. Selain itu aplikasi ini menyediakan berbagai macam kualitas  seperti Mp4 atau WEBN dan banyak lagi. Dengan adanya kecanggihan tersebut aplikasi ini memudahkan user ketika menggunakannya dalam sehari-hari serta aplikasi ini sangat mudah dalam penggunaannya. 


\section{Identifikasi Masalah}
Berdasarkan latar belakang masalah yang ada, maka dapat diidentifikasi menjadi beberapa masalah sebagai berikut:
\begin{enumerate}
\item Bagaimana penerapan fungsi serta algoritma dan pemograman dalam metode download pada Aplikasi ?
\end{enumerate}

\section{Tujuan}
Berdasarkan latar belakang dan identifikasi masalah yang ada, maka akan memberikan tujuan dan manfaat sebagai berikut:
\begin{enumerate}
\item Mengetahui bagaimana sistem yang berjalan dari awal pencarian video hingga mendownload video pada  aplikasi Downloader Mp4 Youtube ( savefrom.net ) dengan bantuan selenium.
\item Mengetahui bagaimana menjalankan sebuah program secara otomatis pada aplikasi Download Mp4 Youtube (savefrom.net) dengan bantuan selenium.
\end{enumerate}

\section{Ruang Lingkup}
Dari data – data yang didapatkan dari proses identifikasi masalah ini maka ruang lingkup dari penelitian ini adalah sebagai berikut:
\begin{enumerate}
\item Sistem cara pengunduhan video dari aplikasi Downloader Mp4 Youtube.
\end{enumerate}

\section{Sistematika Penulisan}
Materi-materi yang tertera pada Laporan Proyek 1 ini dikelompokkan menjadi beberapa sub bab dengan sistematika penyampaian sebagai berikut :
\begin{enumerate}
    \item BAB 1 PENDAHULUAN
    \par
    Pada bab ini berisi latar belakang,identifikasi masalah,tujuan,ruang lingkup dan sistematika penulisan.
    
    \item BAB 2 LANDASAN TEORI
    \par
    Pada bab ini berisi tentang teori yang berupa pengertian atau definisi sistem penggunakan aplikasi Downloader Mp4 Youtube(savefrom.net).
    
    \item  BAB 3 ANALISIS DAN PERANCANGAN
    \par
    Pada bab ini akan dijelaskan function algoritma yang digunakan pada program / sistem aplikasi dan akan dijelaskan bagaimana alur kerja program aplikasi yang akan digambarkan dalam bentuk flowmap sehingga pembaca akan lebih mudah mengerti analisis ini.
    
    \item BAB 4 IMPLEMENTASI
    \par
    Bab ini menjelaskan tentang implementasi yang terdiri dari alat pendukung serta aplikasi pendukung ,lingkungan implementasi, tampilan antar muka, dan petunjuk pemakaian. Bab ini juga membahas tentang pembahasan hasil pengujian.
    
    \item BAB 5 KESIMPULAN DAN SARAN
    \par
    Bab ini berisi kesimpulan dan saran yang berkaitan dengan analisis yang telah dituliskan pada bab sebelumnya sehingga pembaca lebih mudah mengerti hasil dari analisis yang telah dilakukan dan penulis juga dapat memberikan saran di bab ini.

\end{enumerate}